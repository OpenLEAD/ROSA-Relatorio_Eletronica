% %******************************************************************************
% % % introduction.tex %
% %******************************************************************************
% % % Title......: Introduction % % Author.....: GSCAR-DFKI % % Started....: Nov
% 2013 % % Emails.....: alcantara@poli.ufrj.br elael@poli.ufrj
% renan028@gmail.com % % Address....: Universidade Federal do Rio de Janeiro %  
%            Caixa Postal 68.504, CEP: 21.945-970 %              Rio de Janeiro,
% RJ - Brasil.
% %
% %******************************************************************************


% %******************************************************************************
% % SECTION - Eletronica
% %******************************************************************************

\section[Proposta 1 – Placa com Microcontrolador e Gateway Ethernet]{Proposta 1
– Placa com Microcontrolador e \\Gateway Ethernet}

\subsection{Arquitetura da Eletrônica Proposta 1 - versão 1}
A eletrônia é composta pelos seguintes dispositivos:
% Gizele pediu para descrição dos compontes, porém foi julgado inoportuno.
\begin{itemize}
  \item Dois Encoders da IFM: 24V e interface \textit{Controller Area Network}
  (CAN).
  Datasheet em Anexo 1.
  \item Um sonar Super SeaKing da Tritech: 24V e interface RS232. Data\-sheet em
  Anexo 1
  \item Um Sistema Pan \& Tilt da Kongsberg: 24V e interface RS232. Data\-sheet
  em Anexo 1.
  \item Dois sensores Indutivos da Pepperl-Fuchs: 24V e saída ana\-lógico.
  Data\-sheet em Anexo 1.
  \item Um sensor de inclinação da IFM: 24V e saída analógica.
  Data\-sheet em Anexo 1.
  \item Um sensor de pressão da Velki: 24V e saída RS485. Datasheet em Anexo 1.
\end{itemize}

A placa com microcontrolador deve ter disponível todas as alimentações elétricas
e interfaces de comunicação descritas acima, além de saída Ethernet para
comunicação com a base. Na figura~\ref{placa}, pode ser observado o modelo 3D da
placa. Na figura~\ref{com_placa} e figura~\ref{alimentacao_placa}, são
representados os diagramas de interfaces de comunicação e alimentação elétrica,
respectivamente, do sistema. A seguir, será feita uma breve explicação dos
principais componentes da placa.

O microcontrolador AT90CAN64 será responsável pelo monitoramento e controle da
alimentação elétrica de todos os dispositivos, além de ser o respon\-sável pela
comunicação CAN (Controller Area Network) com o Encoder.

A ponte de ligação (gateway) Ethernet SR01E12 possui interfaces de comunicação
UART e analógicas. Des\-sa forma, diversos dispositivos podem se conectar ao
gateway atra\-vés de chips MAX232 ou MAX485, que realizam a conversão RS232 ou RS485 para UART, respectivamente.

\begin{figure}[H]
\centering
\includegraphics[width=1\columnwidth]{figs/eletronica/placav1.png}
\caption{Placa 3D}
\label{placa}
\end{figure}

\begin{figure}[H]
\centering
\includegraphics[width=1\columnwidth]{figs/eletronica/alimv1.png}
\caption{Diagrama de Alimentações}
\label{alimentacao_placa}
\end{figure}

\begin{figure}[H]
\centering
\includegraphics[width=1\columnwidth]{figs/eletronica/comv1.png}
\caption{Diagrama de Comunicação}
\label{com_placa}
\end{figure}

A placa com microcontrolador é uma solução de baixo custo, porém exige maior
tempo de execução. Há a necessidade de fabricação, montagem, testes elétricos e
lógicos da placa e programação de microcontrolador para gerenciamento de cada
interface, como o protocolo CANOpen (protocolo de comunicação para camada de
usuário no modelo ISO/OSI com alto grau de flexibilidade para configuração, e
utiliza o CAN como camada de transporte).

A eletrônica deverá ser acoplada à viga pescadora, logo deverá ser construída
uma estrutura mecânica à prova d’água para esta solução. Conectores e emendas
deverão ser à prova d'água. Os conectores serão do tipo SEA CON WET-CON por já
terem sido estudados e utilizados em outros projetos do Laboratório de Controle
e Automação, Engenharia de Aplicação e Desenvolvimento (LEAD), ver Datasheet em
Anexo 1 - Conectores. Nesta versão, serão utilizados:
\begin{itemize}
  \item Dois conectores de 3 pinos (sensores indutivos);
  \item Um conector de 7 pinos (Pan Tilt);
  \item Três conectores de 5 pinos (encoders e sonar);
\end{itemize}
O sensor de inclinação e pressão estão dentro da eletrônica embarcada e,
portanto, não necessitam de conectores à prova d'água.

\subsection{Arquitetura da Eletrônica Proposta 1 - versão 2}
Após a primeira viagem técnica para Jirau em Maio/2014, houve algumas alterações
na arquitetura da eletrônica. A nova arquitetura da eletrônia é composta pelos
seguintes dispositivos:
\begin{itemize}
  \item Um sonar Super SeaKing da Tritech: 24V e interface RS232. Data\-sheet em
  Anexo 1
  \item Um Sistema Pan \& Tilt da Kongsberg: 24V e interface RS232. Data\-sheet
  em Anexo 1.
  \item Três sensores Indutivos da Pepperl-Fuchs: 24V e saída ana\-lógico.
  Data\-sheet em Anexo 1.
  \item Quatro sensor de inclinação da IFM: 24V e saída analógica.
  Data\-sheet em Anexo 1.
  \item Um sensor de pressão da Velki: 24V e saída RS485. Datasheet em Anexo 1.
\end{itemize}
Os sensores de inclinação acrescentados podem substituir os encoders, em caso de
restrição de acoplamento mecânico na viga pescadora. 
Os sensores de inclinação serão acoplados da mesma
maneira que os sensores indutivos

A placa com microcontrolador é semelhante à desenvolvida na versão 1,
acrescida de quatro conectores para os sensores de inclinação e mais um
conector para o sensor indutivo extra. 

O microcontrolador AT90CAN64 será responsável pelo monitoramento e controle da
alimentação elétrica de todos os dispositivos, além de interpretar os dados dos
quatro sensores de inclinação e se comunicar com o gateway Ethernet. 

Os sensores de inclinação adicionais podem substituir os encoders, em caso de
restrição de acoplamento mecânico na viga pescadora. Eles têm saída
analógica, as quais são interpretadas pelo microcontrolador pelos canais
ADC (conversor analógico-digital).

O sensor indutivo adicional será utilizado para detectar a posição da chave de
operação, sendo possível alertar o operador quando a operação está na fase de
remoção ou inserção de stoplogs.

Duas placas foram desenvolvidas:~\ref{placav21} e ~\ref{placav22}, onde a segunda apresenta
amplificadores para condicionamento dos sinais analógicos provenientes dos
sensores de inclinação a fim de reduzir ruídos. Na figura~\ref{com_placav2} e
figura~\ref{alimentacao_placav2}, são representados os diagramas de interfaces de comunicação e alimentação elétrica,
respectivamente, do sistema.

\begin{figure}[H]
\centering
\includegraphics[width=1\columnwidth]{figs/eletronica/placav21.png}
\caption{Placa 3D - versão com sensores de inclinação sem condicionamento de
sinal}
\label{placav21}
\end{figure}

\begin{figure}[H]
\centering
\includegraphics[width=1\columnwidth]{figs/eletronica/placav22.png}
\caption{Placa 3D - versão com sensores de inclinação com condicionamento de
sinal}
\label{placav22}
\end{figure}

\begin{figure}[H]
\centering
\includegraphics[width=1\columnwidth]{figs/eletronica/alimentacao_placav2.pdf}
\caption{Diagrama de Alimentações}
\label{alimentacao_placav2}
\end{figure}

\begin{figure}[H]
\centering
\includegraphics[width=1\columnwidth]{figs/eletronica/com_placav2.pdf}
\caption{Diagrama de Comunicação - versão 2}
\label{com_placav2}
\end{figure}

A eletrônica desta versão 2 também será acoplada ao Lifting Beam. Os
conectores utilizados nesta versão serão:
\begin{itemize}
  \item Seis conectores de 3 pinos (sensores indutivos e inclinômetros);
  \item Um conector de 7 pinos (Pan Tilt);
  \item Um conector de 5 pinos (sonar);
\end{itemize}
Um sensor de inclinação e o sensor de pressão estão dentro da eletrônica
embarcada e, portanto, não necessitam de conectores à prova d'água.

\subsection{Arquitetura da Eletrônica Proposta 1 - versão 3}
O módulo Ethernet, na arquitetura da eletrônica das versões anteriores, tem a
função de gerenciar quatro comunicações seriais. A transição entre as quatro
comunicações seriais (uC, Sonar, Pan \& Tilt e Profundímetro) seria feita por
software, o que pode resultar em perda de dados, já que apenas duas são
ativadas simultaneamente, limitação do módulo. A solução para isso foi
o desenvolvimento de uma nova placa eletrônica com dois módulos Ethernet e um
switch Ethernet para integrar todos os dados.

Na nova placa, a saída analógica dos inclinômetros passam pelo condicionamento
de sinal e são analisadas diretamente por um dos módulos Ethernet. O esquemático
da versão 3 pode ser visto na figura~\ref{esquematicov3} e o modelo 3D na
figura~\ref{placav3}.

\begin{figure}[H]
\centering
\includegraphics[width=1\columnwidth]{figs/eletronica/esquematicov3.pdf}
\caption{Esquemático da Versão 3}
\label{esquematicov3}
\end{figure}

\begin{figure}[H]
\centering
\includegraphics[width=1\columnwidth]{figs/eletronica/placav3}
\caption{Layout 3D da placa Versão 3}
\label{placav3}
\end{figure}

Segue lista de conectores e cabos da SEACON WET-CON para o housing da
eletrônica (os conectores são representados no manual do usuário):
\begin{itemize}
  \item BH-3-MP com cabo de 2m. Quantidade: 6;
  \item BH-8-MP com cabo de 2m. Quantidade: 1;
  \item BH-6-MP com cabo de 2m. Quantidade: 2;
  \item BH-2-MP. Quantidade: 1;
  \item BH-3-FP. Quantidade: 6;
  \item BH-8-FP. Quantidade: 1;
  \item BH-6-FP. Quantidade: 2;
  \item BH-2-FP com cabo de 2m. Quantidade: 1;
  \item DC-3-FS. Quantidade: 6
\item DC-8-FS. Quantidade: 1	
\item DC-6-FS. Quantidade: 2
\item DC-2-FS. Quantidade: 1
\item DC-3-MP. Quantidade: 6
\item DC-8-MP. Quantidade: 1
\item DC-6-MP. Quantidade: 2
\item DC-2-MP. Quantidade: 1   
\end{itemize}

\subsection{Arquitetura de Software Proposta 1 - versão 1}
O sistema se divide em três grandes blocos que serão encapsulados separadamente
e se comunicarão entre si. A primeira parte consiste no
sistema de eletrônica embarcada composta pelos sensores e um equipamento de
roteamento dos dados (Na figura \ref{fig:FL:1} ela está representada segundo as
raias: \emph{Sensores}, \emph{Conversores UART} e \emph{Interface de Telemetria}
), a segunda parte consiste no sistema de gerenciamento e processamento de dados em terra (raia \emph{Computação em
Terra}) e, finalmente, o último bloco consiste na camada de interface
homem-máquina , que consiste em um Tablet com sistema operacional Android (raia
\emph{Interface com o usuário}).

\afterpage{
\begin{figure}
\centering
\includegraphics[width=1\linewidth,keepaspectratio]{figs/software/LogicoFisico/EsqLogicoFisico1n.pdf}
\caption{Relação entre os componentes físicos e as divisões lógicas da
proposta 1.}
\label{fig:FL:1}
\end{figure}
}

Cada sensor deverá possuir um driver para a interface entre o equipamento físico
e camada de software, isto é, os encoders, inclinômetro, os sensores indutivos e
o sensor de pressão possuirão drivers dedicados para a leitura de dados, feita
através de uma conexão Ethernet para a placa que interconecta os sensores. O
sonar e o módulo Pan \& Tilt também possuirão drivers próprios para a aquisição
de dados e controle.

Nesta proposta, após a aquisição de dados, a placa embarcada realizará o
roteamento e envio de todos os dados para o computador em terra por meio do
driver do módulo Ethernet.  Os dados transmitidos pela eletrônica embarcada são
separados em dois grandes grupos: referentes à Monitoração e os referentes à
Visualização Sonar.

No computador localizado em terra, figura \ref{fig:ES:1}, o componente de
software responsável pela monitoração irá processar e conformar os dados
provenientes dos sensores utilizados para o monitoramento das operações de
inserção e remoção (encoders, inclinômetro, sensores indutivos e sensor de
pressão). Os dados provenientes do sonar devem ser integrados com a posição do
elemento PanTilt, no componente Sonar-PanTilt, para que sejam consistentes e
completos.  O módulo de Reconstrução 3D é responsável, então, por traduzir os
dados processados pelo componente anterior em uma visualização inteligível para
o ser humano.  Um componente de segurança também é adicionado para monitorar a
correta utilização do sonar (apenas embaixo d’água), conferindo uma maior
robustez ao sistema.

\begin{figure}[H] 
\centering
\includegraphics[width=\textwidth,height=\textheight,keepaspectratio]{figs/software/EstrutSoft/prop1_soft_2.pdf}
\caption{Interconexões entre os componentes de software da proposta 1.}
\label{fig:ES:1}
\end{figure}

Ambos os componentes de processamento e conformação de dados, Monitoração e
Visualização Sonar, irão se comunicar com o Tablet com sistema operacional
Android por meio do componente Transmissão de dados – Tablet, enviando os dados
processados e recebendo os comandos de controle.  A interface homem máquina
consiste em um aplicativo cuja finalidade é realizar a interface do sistema com
o usuário, possibilitando uma correta e fácil visualização de todas as
informações pertinentes do sistema e suas operações.