% %******************************************************************************
% % % introduction.tex %
% %******************************************************************************
% % % Title......: Introduction % % Author.....: GSCAR-DFKI % % Started....: Nov
% 2013 % % Emails.....: alcantara@poli.ufrj.br elael@poli.ufrj
% renan028@gmail.com % % Address....: Universidade Federal do Rio de Janeiro %  
%            Caixa Postal 68.504, CEP: 21.945-970 %              Rio de Janeiro,
% RJ - Brasil.
% %
% %******************************************************************************


% %******************************************************************************
% % SECTION - Eletronica
% %******************************************************************************

\section[Proposta 1 – Placa com Microcontrolador e Gateway Ethernet]{Proposta 1
– Placa com Microcontrolador e \\Gateway Ethernet}

\subsection{Arquitetura da Eletrônica Proposta 1}
A eletrônia é composta pelos seguintes dispositivos:
\begin{itemize}
  \item Dois Encoders da IFM: 24V e interface CAN. Datasheet em Anexo 1.
  \item Um sonar Super SeaKing da Tritech: 24V e interface RS232. Data\-sheet em
  Anexo 1
  \item Um Sistema Pan \& Tilt da Kongsberg: 24V e interface RS485. Data\-sheet
  em Anexo 1.
  \item Dois sensores Indutivos da Pepperl-Fuchs: 24V e saída ana\-lógico.
  Data\-sheet em Anexo 1.
  \item Um sensor de inclinação da Pepperl-Fuchs: 24V e saída analógica.
  Data\-sheet em Anexo 1.
  \item Um sensor de pressão da Velki: 24V e saída RS485. Datasheet em Anexo 1.
\end{itemize}

A placa com microcontrolador deve ter disponível todas as alimentações elétricas
e interfaces de comunicação descritas acima, além de saída Ethernet para
comunicação com a base. Na figura~\ref{placa}, pode ser observado o modelo 3D da
placa. Na figura~\ref{alimentacao_placa} e figura~\ref{com_placa}, são
representados os diagramas de interfaces de comunicação e alimentação elétrica,
respectivamente, do sistema. A seguir, será feita uma breve explicação dos
principais componentes da placa.

O microcontrolador AT90CAN64 será responsável pelo monitoramento e controle da
alimentação elétrica de todos os dispositivos, além de ser o respon\-sável pela
comunicação CAN com o Encoder.

O gateway Ethernet SR01E12 possui interfaces UART e analógicas. Des\-sa forma,
diversos dispositivos podem se conectar ao gateway atra\-vés de chips MAX232 ou
MAX485, que realizam a conversão RS232 ou RS485 para UART, respectivamente.

\begin{figure}[H]
\centering
\includegraphics[width=1\columnwidth]{figs/eletronica/1.png}
\caption{Placa 3D}
\label{placa}
\end{figure}

\begin{figure}[H]
\centering
\includegraphics[width=1\columnwidth]{figs/eletronica/2.png}
\caption{Diagrama de Alimentações}
\label{alimentacao_placa}
\end{figure}

\begin{figure}[H]
\centering
\includegraphics[width=1\columnwidth]{figs/eletronica/3.png}
\caption{Diagrama de Comunicação}
\label{com_placa}
\end{figure}

A placa com microcontrolador é uma solução de baixo custo, porém exige maior
tempo de execução. Há a necessidade de fabricação, montagem, testes elétricos e
lógicos da placa e programação de microcontrolador para gerenciamento de cada
interface, como o protocolo CANOpen.
A eletrônica deverá ser acoplada ao Lifting Beam, logo deverá ser construída uma
estrutura mecânica à prova d’água para esta solução.

\subsection{Arquitetura de Software Proposta 1}
O sistema se divide em três grandes blocos que serão encapsulados separadamente
e se comunicarão entre si, figura \ref{fig:FL:1}. A primeira parte consiste no
sistema de eletrônica embarcada composta pelos sensores e um equipamento de
roteamento dos dados (raias \emph{Sensores}, \emph{Conversores UART} e
\emph{Interface de Telemetria} da figura), a segunda parte consiste no sistema
de gerenciamento e processamento de dados em terra (raia \emph{Computação em
Terra}) e, finalmente, o último bloco consiste na camada de interface
homem-máquina , que consiste em um Tablet com sistema operacional Android (raia
\emph{Interface com o usuário}).

\afterpage{
\begin{landscape}
\begin{figure}
\centering
\includegraphics[width=1\linewidth,keepaspectratio]{figs/software/LogicoFisico/EsqLogicoFisico1n.pdf}
\caption{Relação entre os componentes físicos e as divisões lógicas da
proposta 1.}
\label{fig:FL:1}
\end{figure}
\end{landscape}
}

Cada sensor deverá possuir um driver para a interface entre o equipamento físico
e camada de software, isto é, os encoders, inclinômetro, os sensores indutivos e
o sensor de pressão possuirão drivers dedicados para a leitura de dados, feita
através de uma conexão Ethernet para a placa que interconecta os sensores. O
sonar e o módulo PanTilt também possuirão drivers próprios para a aquisição de
dados e controle.

Nesta proposta, após a aquisição de dados, a placa embarcada realizará o
roteamento e envio de todos os dados para o computador em terra por meio do
driver do módulo Ethernet.  Os dados transmitidos pela eletrônica embarcada são
separados em dois grandes grupos: referentes à Monitoração e os referentes à
Visualização Sonar.

No computador localizado em terra, figura \ref{fig:ES:1}, o componente de
software responsável pela monitoração irá processar e conformar os dados
provenientes dos sensores utilizados para o monitoramento das operações de
inserção e remoção (encoders, inclinômetro, sensores indutivos e sensor de
pressão). Os dados provenientes do sonar devem ser integrados com a posição do
elemento PanTilt, no componente Sonar-PanTilt, para que sejam consistentes e
completos.  O módulo de Reconstrução 3D é responsável, então, por traduzir os
dados processados pelo componente anterior em uma visualização inteligível para
o ser humano.  Um componente de segurança também é adicionado para monitorar a
correta utilização do sonar (apenas embaixo d’água), conferindo uma maior
robustez ao sistema.

\begin{figure}[H] 
\centering
\includegraphics[width=\textwidth,height=\textheight,keepaspectratio]{figs/software/EstrutSoft/prop1_soft_2.pdf}
\caption{Interconexões entre os componentes de software da proposta 1.}
\label{fig:ES:1}
\end{figure}

Ambos os componentes de processamento e conformação de dados, Monitoração e
Visualização Sonar, irão se comunicar com o Tablet com sistema operacional
Android por meio do componente Transmissão de dados – Tablet, enviando os dados
processados e recebendo os comandos de controle.  A interface homem máquina
consiste em um aplicativo cuja finalidade é realizar a interface do sistema com
o usuário, possibilitando uma correta e fácil visualização de todas as
informações pertinentes do sistema e suas operações.
