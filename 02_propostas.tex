%%******************************************************************************
%%
%% introduction.tex
%%
%%******************************************************************************
%%
%% Title......: Introduction
%%
%% Author.....: GSCAR-DFKI
%%
%% Started....: Nov 2013
%%
%% Emails.....: alcantara@poli.ufrj.br elael@poli.ufrj renan028@gmail.com
%%
%% Address....: Universidade Federal do Rio de Janeiro
%%              Caixa Postal 68.504, CEP: 21.945-970
%%              Rio de Janeiro, RJ - Brasil.
%%
%%******************************************************************************


%%******************************************************************************
%% SECTION - Eletronica
%%******************************************************************************

\section{Propostas de soluções para a arquitetura da eletrônica do Projeto ROSA}

Foram desenvolvidas três soluções para a arquitetura da eletrônica, que podem
ser classificadas quanto à simplicidade, custo e rapidez de execução.

A primeira solução consiste em projetar uma placa que pudesse realizar o
controle da alimentação dos dispositivos da eletrônica, monitoramento elétrico
de corrente e voltagem, e gerenciar os dispositivos, com todas as interfaces de
comunicação do sistema. O processamento do sonar será realizado por um PC na
base. Os componentes da eletrônica necessários para a proposta são de baixo
custo, porém há complexidade de software e eletrônica em relação às outras
soluções, o que impacta em uma maior demora da solução.

A segunda proposta incide na utilização de um PC embarcado para o processamento
de sinal e gerenciamento de todas as interfaces necessárias para o gerenciamento
dos dispositivos. O PC embarcado se comunica com a base por Ethernet, onde
haverá um roteador para estabelecer a comunicação com o Tablet.
Esta solução necessita de um PC104 embarcado, com proteção mecânica desenvolvida
pela equipe, ou uma eletrônica importada protegida. Do ponto de vista da
eletrônica, a solução tem com PC104 apresenta custo intermediário, é simples e
de baixo tempo de execução, porém há a complexidade mecânica. Por outro lado, a
eletrônica importada é de alto custo, mas aproxima-se mais de um produto final.

A terceira proposta é a utilização de dois PCs, de forma que haja processamento
tanto na eletrônica embarcada, quanto na base. O custo desta solução é alto,
porém apresenta grande simplicidade e rápido tempo de execução.